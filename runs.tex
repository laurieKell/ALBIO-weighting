\subsubsection*{Runs Test}

Analysis of residuals is a common way to determine a model’s goodness-of-fit \citep{Cox1968general}, since  non-random patterns in the residuals may indicate model misspecification, serial correlation in sampling/observation error, or heteroscedasticity. 

When inspecting residuals, however, there is a danger of hypothesis fishing and if multiple true hypotheses are tested it is likely that some of them will be rejected. Therefore it is valuable to reserve part of the data for validation, so that a pattern’s significance is not tested on the same data set which suggested the pattern.


If the process of interest shows only random variation, the data points will be randomly distributed around the median. Random meaning that we cannot know if the next data point will fall above or below the median, but that the probability of each event is 50\%, and that the data points are independent. Independence means that the position of one data point does not influence the position of the next data point, that is, data are not auto-correlated. If the process shifts, these conditions are no longer true and patterns of non-random variation may be detected by statistical tests. Various statistics exist to evaluate residuals and nonparametric tests for randomness in a time-series include: the runs test, the sign test, the runs up and down test, the Mann-Kendall test, and Bartel’s rank test.

Non-random variation may present itself in several ways. If the process centre is shifting due to improvement or degradation we may observe unusually long runs of consecutive data points on the same side of the median or that the graph crosses the median unusually few times. The length of the longest run and the number of crossings in a random process are predictable within limits and depend on the total number of data points in the run chart \citep{anhoj2015diagnostic}.

A shift signal is present if any run of consecutive data points on the same side of the median is longer than the prediction limit, round(log2(n) + 3). Data points that fall on the median do not count, they do neither break nor contribute to the run \cite{schilling2012surprising}. A crossings signal is present if the number of times the graph crosses the median is smaller than the prediction limit, qbinom(0.05, n - 1, 0.5) \citep{chen2010impacts}. n is the number of useful data points, that is, data points that do not fall on the median. The shift and the crossings signals are based on a false positive signal rate around 5\% and have proven useful in practice.