\subsection*{Methods}

The performance of multimodel ensembles depends on the weights given to the different models in postprocessing. Although still common practice, equal weighting of all models in a grid or ensemble may result in biased advice if too much weight is assigned to models that fit the data poorly or have low have poor prediction skill \citep{kell2020yft}.

We therefore explore a number of different potential weighting schema, and methods for selecting and rejecting OM scenarios, and compare equal weighting (EW), with simple skill-based weighting (SW) \citep{casanova2009weighting}.

\subsubsection*{Weighting}

%Considering alternative structures and datasets means that it difficult to select models using metrics such as AIC. Therefore retrospective analysis is commonly used to evaluate the stability of stock assessment estimates of model estimates such as stock biomass and exploitation level. Stability is measured using Mohn's $\rho$ a measure of bias. Shrinking estimates of stock status in the last year to the recent mean can help reduce Mohn's $\rho$. Shrinkage, in statistics, however, is used to reduce mean squared error (MSE), at the expense of bias. The use of model based quantities, however, means that bias can not be quantified, and may result in forecasts having little prediction skill. We therefore extend retrospective analyses to include prediction. 

\begin{description}
\item[Expert] Stakeholder concerns and expert opinion, assigned “a-priori”, without consideration of model fit.
 \begin{itemize}
    \item How were scenarios selected?, e.g. through an elicitation process \citep{leach2014identification}?
\end{itemize}
\item[Convergence] Model convergence criteria of the estimation algorithm. 
\begin{itemize}
    \item max gradient? This is normally achieved by jittering, however this is difficult to do in this case where 1440 grids are run. Therefore we explore what factors result in poor convergence.
\end{itemize}
\item[Diagnostics] Reliability of the model based on residual diagnostics
 \begin{itemize}
    \item Use runs tests to check for randomness in the residuals 
\end{itemize}
 \item[Fit] The fit of the model to the data
 \begin{itemize}
    \item Likelihood, we do not use AIC due to problems in weighting and the fact that since the number of parameters only changes for 1 scenario the AIC and likelihoods are equivalent. Using the likelihood also allows us to look at data components.
\end{itemize}
\item[Plausible parameters] The plausibility of the estimates of the parameters representing the dynamics
 \begin{itemize}
    \item  $r$, $K$ and $p$, i.e. production functions
\end{itemize}
\item[Plausible results] Time series dynamics and process error.
\begin{itemize}
    \item ACF of biomass
    \item clockwise SP
    \item var(SP)
\end{itemize}
\item[Stability] Retrospective
https://www.overleaf.com/project/5f06c2b7d4ece70001d5c362\begin{itemize}
    \item traditional Mohn's $\rho$
\end{itemize}
\item[Prediction Skill] Model outputs
 \begin{itemize}
    \item 3 year ahead Mohn's $\rho$ 
\end{itemize}
\item[Prediction Skill] Model free
\begin{itemize}
    \item MASE for CPUE
\end{itemize}
\end{description}

\input{runs}
\input{Hindcast}
\subsubsection*{Generating delta-MVLN Kobe posteriors}

Generate Kobe posteriors from a MVLN distribution requires the means and the variance-covariance matrix (VCM) of $log(SSB/SSB_{MSY})$ and $log(F/F_{MSY})$. Let $u = SSB/SSB_{MSY}$ and $v = F/F_{MSY}$  and $x = log(u)$ and $y = log(v)$ , then the $VCM$ has the form:
    			
\begin{equation}
VCM_{x,y} =
\begin{pmatrix}
\sigma^2_x & cov_{x,y}  \\
cov_{x,y} & \sigma^2_y
\end{pmatrix}
\end{equation*}

where  is the variance of x,  is the covariance of y and  is the covariance of x and y.  The quantities that can be directly extracted from Stock Synthesis are: (1) MLEs, asymptotic standard errors (SE) and correlation of $SSB/SSB_{MSY}$  and $F/F_{MSY}$. 
The construction of the  therefore requires to conduct a few normal to lognormal transformations. First, we approximate  and  as:

\begin{equation}
\sigma^2_x = \disp log\left(1+\left(\frac{SE_u}{u}\right)^2\right)  
\quad 
\end{equation}

and

\begin{equation}
\sigma^2_x = \disp log\left(1+\left(\frac{SE_v}{v}\right)^2\right)  
\quad 
\end{equation}

where  and  is the asymptotic standard error estimate for $u = SSB/SSB_{MSY}$ and $v = F/F_{MSY}$. Second, the covariance of x and y can then be approximated on log-scale by:

\begin{equation}
COV_{x,y} = \disp log \right{1+ \rho_{u,v} \sqrt{\sigma^2_x\sigma^2_y}\right \quad 
\end{equation}

where  donates the correlation of u and v.
To generate the desired KPD for $SSB/SSB_{MSY}$  and $F/F_{MSY}$, we use a multivariate random generator, available in the R package ‘mvtnorm’, to obtain a large number (nsim = 10,000) of x and y pairs, such that

\begin{equation}
kobe_{x,y} = \disp MVN{\mu_{x,y},VCM_{x,y}) 
\quad 
\end{equation}

where  is the vector of the MLEs x and y. The joint MVLN distribution of $-SSB/SSB_{MSY}$  and $-F/F_{MSY}$ is then obtained as the exponential of $kone_{x,y}$.



There are a variety of frequentist model-weighting strategies ranging from giving all models in S a weight related to the AIC (Akaike information criterion) or BIC (Bayesian information criterion) of each model, to an interpolation between two extreme cases. We present an AIC-based weighting strategy called smooth AIC weights that was first presented by Buckland et al. (1997). In smooth AIC weighting, the weights are given by:  
wM=exp(−(1/2)AICM)∑M′∈Sexp(−(1/2)AICM′).
(6)

It is suggested to subtract the minimum AIC from each model AIC to avoid numerical issues when taking exponents.

%In age-structured models, there is an implicit production function, and changes in productivity can occur due to process error modelled as variability in recruitment and selection pattern. In the biomass-dynamic models with an explicit production function, process error is modelled explicitly.

In age-structured models, density dependence is mainly accounted for by the stock-recruitment relationship. Cury et al. (2014), however, showed that in most cases the stock-recruitment relationship used to estimate productivity and determine reference points, has poor estimation/predictive power and the environment has a larger effect on productivity, a result confirmed by other studies (e.g. Szuwalski et al., 2015, 2019; Free et al., 2019), and observed 100 years ago by Hjort (1914). Whereas in ICCAT assessments growth, maturation and natural mortality are assumed not to have varied despite the significant changes in the environment and stock biomass seen.

Hilborn (2001) therefore recommended looking at patterns of change in surplus production (SP) since these may contain evidence of changes in the growth and mortality components of production, which are typically not represented in models currently used for stock assessment and management. 

Walters et al. (2008) argued that plotting of surplus production (SP) against biomass (B) should be one of the basic pieces of information presented in all stock assessments since the plots provide a check on whether there has been non-stationarity in the annual surplus production, i.e. whether similar B levels have exhibited similar SP at different historical times. This is important for management as it checks whether predictions of changes in biomass (Bt+1−Bt) can be made reliably based on catch and Bt. Plots of SP v B therefore provide a summary of stock performance and include effects not necessarily included in stock assessment models.

The effects not included in the production function used to predict SP can be modelled by a process error term # t.

				$B_{t+1} = B_t − C_t + SP_t + ε_t$

Process error on biomass can account for model structural uncertainty as well as natural variability of stock biomass due to stochasticity in recruitment, natural mortality, 4growth, and maturation (Francis and Hilborn, 2011; Meyer and Millar, 1999; Thorson et al., 2015).

We, therefore, examine the relationships between surplus production and biomass . To do this, we estimate annual surplus production as the change in stock size plus catch (i.e. $B_t – B_{t+1} + C_t$). We then plot the resulting time series of S and B to identify patterns of variation in S. The process error was then sampled from SS3 stock trajectories as the difference between the deterministic expectation of biomass and its stochastic realisation, such that:

				$ε_t = SB_{t+1} - (SB_t + SP_t − C_t)$ 

