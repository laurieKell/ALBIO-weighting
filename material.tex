\section*{Material and Methods}

An Operating Model for the albacore tuna (\textit{Thunnus alalunga}) fishery and stock in the Indian Ocean has been developed by the Indian Ocean Tuna Commission (IOTC) to evaluate the performance of alternative Management Procedures (MPs).  The Operating Model (OM) was conditioned using stock synthesis \citep[SS3][]{MethotW2013} based on current best knowledge and data \citep{ptmt2014}. Datasets include records of catches and landings, indices of abundance based on catch per unit (CPUE), and  length and ages compositions from samples. Reformulating model structure, however, is time consuming and so scenarios were based on the choice of parameters for which there is insufficient information in the data to estimate and data weighting. 

\subsection*{Material}

The assesssment partitions the Indian Ocean into four regions, divided latitudinally along the 25$^{\circ}$S parallel and longitudinally along the 75$^{\circ}$E meridian. Aggregated catches (in numbers of fish) for the Japanese and Taiwanese LL fleets, for the 1950-2014 period, are shown. From \citep{LangleyH2016}, figure \ref{fig:map} show the distribution of catches across the areas defined in the assessment. The model includes a total of 11 fisheries, including in this case an aggregaqted Longline fishery for each of the four regions. For a detailed explanation of the data and fleets included in each of these fisheries, please refer to \citep{LangleyH2016}. A new set of standardized CPUE indices has been derived using generalized linear models (GLM) operational from longline catch and effort data provided by Japan, Korea and Taiwan, China. \citep{HoyleKL2016}. The operating model conditioning used the same series as the final runs of the stock assessment \citep{LangleyH2016}, a combined industrial longline series, on each of the four areas, and restricted to the 1979-2014 period (Figure \ref{fig:cpues}). Of these four areas, area 3 is considered to represent the core of the distribution of the stock. The management procedures tested make use of a single CPUE, taken to be that corresponding to area 3.

The distribution of catches by assessment areas are shown in are shown in figure \ref{fig:map}.
   
Fisheries data is in general less informative that would be ideal when it comes to estimating a large number of model parameters, which are often correlated. In the case of the Indian Ocean albacore stock, a number of reasons are limiting our ability to obtain reliable model fits. Problems exists with the data completeness and quality \citep{IOTC2016WPTmT0607}, not limited to but including total catch statistics, length distribution in catches, and biological information. We also depend on our ability to produce sensible indices of changes in abundance in the stock based only on Catch-per-unit-effort data from commercial fleets, where issues of targeting, operating and others are all known to influence the relationship between stock abundance and CPUE, despite recent work on standardization of the longline CPUE series for this stock \citep{oyleKL2016}.

The seven factors currently considered in the structural uncertainty grid for the albacore OM are shown in  table \ref{tab:om})

A common unknown in most stock assessment models, the base case considered in the stock assessment session was supplemented with alternative values of higher and lower M for either all ages, or different for juveniles (ages 0 to 4) and adults (age 5 or older), for a total of five possibilities. Two values were considered for the true variability of recruitment in the population, 0.4 and 0.6. Three values for the steepness (h) of the stock-recruitment relationship are being used: 0.7, 0.8, and 0.9. The Beverton and Holt stock-recruit model implemented. Four values for the coefficient of variation in the CPUE series were included: 0.2, 0.3, 0.4 and 0.5. Three values were used for the relative weight of length sampling data in the total likelihood, through changes in the effective sampling size parameter, of 20, 50 and 100. This alters the relative weighting of length samples and CPUE series in informing the model about stock dynamics and the effects of fishing at length. Two scenarios were considered for the effective catchability of the CPUE fleet. On the first one it was assumed that the fleet had not improved its ability to fish for albacore over time, or that any increase had been captured by the CPUE standardization process. An alternative scenario considered a 2.5\% increase in catchability by correcting the CPUE index to reflect this.  Two possible functional forms for the selectivity of the CPUE LL fleet were considered: a logistic function (Log), where selectivity stays at the maximum level, or double normal (DoNorm), where selectivity drops at some point in the age range.

