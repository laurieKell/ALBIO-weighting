\section*{Introduction}


The adoption of the Precautionary Approach to fisheries management \citep[PA,][]{garcia1996precautionary} requires a formal consideration of uncertainty. This had led to the use of Management Strategy Evaluation (MSE) is to develop robust advice that can still meet management objectives despite uncertainty. When conducting MSE Operating Models (OMs) are used to represent alternative hypotheses about the dynamics of the system. The most common procedure when is to fit to the available data on the basis of some statistical criterion, such as a Maximum Likelihood. The aim of this conditioning process is to reject OMs that do not fit the data satisfactorily, and are inconsistent with the actual dynamics.  

 There has been a trend when conducting MSE to use integrated assessment models for conditioning, as this allows different sources of data to be combined into a single model by a joint likelihood  \citep[e.g.][]{doubleday1976least,fournier1982general,maunder2013review}, as this allows scientists seek to use the models to capture all knowledge about stock size and productivity \citep{hilborn2003state}. Problems remain, however, including a lack of information on spatial and temporal processes that may affect stationarity,  density dependence, and conflicts between datasets. Misspecification of key parameters or assumptions in integrated stock assessment models can strongly impact the estimates of quantities of management interest, such as stock depletion and biomass at maximum sustainable yield \citep{mangel2013perspective}. Therefore, the impact of uncertainty about resource dynamics is commonly evaluated by the use of a grid-based design that considering alternative model structure, datasets and parameters \citep{sharma2020mse}. 
 
 Integrated models are commonly used to conditioning Operating Models (OMs) as part of Management Strategy Evaluation (MSE). This involves fitting an OM of the resource dynamics to the available data on the basis of some statistical criterion, such as a Maximum Likelihood.  The aim of conditioning is to reject OMs that do not fit the data satisfactorily, and are consequently inconsistent with the actual situation observed and therefore implausible. Therefore, when conditioning OMs the intention is not to find a "best assessment" but a limited set of OMs with high plausibility, which include the most important uncertainties in the model structure, parameters, and data. Plausibility may be estimated formally based on some statistical approach, or specified based on expert judgement, and can be used to weight performance statistics when integrating over results for different scenarios (OMs). The methods developed in the cookbook can be used to do this. 
 
 To date, however, little effort has gone into weighting of the different models, e.g. based on ensemble modelling. Advice, however, depends on the weights given to the different models when post processing, and critically the original choice of models. Therefore questions that need to be answered in the conditioning are does the model presents a good fit to the data and is it able to predict the response to fishing \citep{carvalho2020cookbook}. We therefore develop a schema for weighting and potential rejection of stock assessment model scenarios, and compare different metrics to simple skill-based weighting (SW).
 
