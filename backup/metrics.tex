
In stock assessment, a standard diagnostic is to evaluate retrospective bias as proposed by \cite{mohn1999retrospectyive}. The retrospective analysis is conducted by sequentially refitting the model to reduced data sets by removing some recent years' data to see if there are any systematic pattern within a model. The retrospective bias is then evaluated using the so-called Mohn's rho as 

\[
\rho = \disp \sum_{t=T-n}^{T-1} \frac{\hat{y}_{(1:t),t}-\hat{y}_{(1:T),t}}{\hat{y}_{(1:T),t}}, 
\]
where $\hat{y}$ denotes in general a value like estimated biomass, 1+population size, or predicted abundance index, and the value with suffix $\hat{y}_{(1:t^\prime),t}$ means such a value estimated at time $t$ of a full series from 1 to $T$ using a retrospective data window from 1 to $t^\prime (\leq T)$. In this paper, we will use a variant of the original $\rho$ as the mean (average) like 
\begin{equation}
\rho_r = \disp \frac{1}{n} \sum_{t=T-n}^{T-1} \frac{\hat{y}_{(1:t),t}-\hat{y}_{(1:T),t}}{\hat{y}_{(1:T),t}} 
\quad \mbox{[rho for retro-bias]}, 
\end{equation}
This metric is an average of relative differences at the final time of each window. Therefore it is a measure of relative retrospective `bias' (scale-free) in a statistical sense. The metric tends to be applied not on the log but the original scale because both the directions of positive and negative biases are regarded as being equivalent. 

Hindcasting, is a form of retrospective cross-validation, and therefore an extension of retrospective analysis which projects several steps forward beyond the retrospective data window to quantify the prediction skill of a model. Theoretically, the projection period is to the end of the historical time period. However, in practice, the step size is one or several years ahead reflecting the requirements for robust management advice, and considering non-small process stochasticity in fishery population dynamics and non-ignorable extents of observation uncertainty. For evaluating prediction skill, we propose several metrics for model-dependent and model-free validations.

We  define `retro-period' and `hc-period' as `the period of shrunken data set for retrospective model fitting' and `future time period with a certain projection step (say $S \geq 1$) for hindcasting after retro-period''. And let $\hat{y}_{(1:t),t+S}$ be an projected value at time $t+S$ in an hc-period based on the conditioned model with data in a retro-period $(1,t)$. 

\vspace{0.2cm} \noindent
{\it Modified Mohn's rho for prediction bias and absolute error:}\\
\begin{equation}
\rho_p = \disp \frac{1}{n-S+1} \sum_{t=T-n}^{T-S} 
\frac{\hat{y}_{(1:t),t+S}-\hat{y}_{(1:T),t+S}}{\hat{y}_{(1:T),t+S}} 
\ \mbox{[rho for projection-bias]}
\end{equation} 

This is a simple extension of Mohn's rho to evaluate the prediction skill of a model because all the values are produced under the model assumption. In this sense, it is a model-dependent consistency check of prediction skill. To evaluate the absolute prediction error for the following can be used
\begin{equation}
|\rho_p| = \disp \frac{1}{(n-S+1)} \sum_{t=T-n}^{T-S}
\frac{\left| \hat{y}_{(1:t),t+S}-\hat{y}_{(1:T),t+S} \right|}{\hat{y}_{(1:T),t+S}}. 
\ \mbox{[rho for projection-absolute-error]}
\end{equation} 
