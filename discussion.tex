\section*{Discussion}


\subsection*{Summary}

 \begin{description}
   \item[Scenarios] overfitted and biased
   \item[Important] to consider model error, simple prediction skill and model-free hindcast
   \item[M] has the biggest impact
   \item[Conflict] between Length Comps and CPUE.
   \item[Base case and main effects] many main effects had poor prediction skill, these could have been excluded. 
   \item[Interactions?] yes for M and ESS/CPUE weighting.
\end{description}
   
\begin{description}
    \item[Screening procedure]~
    \begin{itemize}
        \item The screening process includes checking goodness of fit diagnostics and prediction skill. I.e. use runs test to look for goodness of fit and data conflicts, then prediction residuals and MASE to check for overfitting, and a model-based hindcast to estimate simple prediction skill.
        \item First propose and screen a base case, then main effects and then interactions. 
       \item The procedure can be used for skilled based weighting as well as screening.
    \end{itemize}
  \item[Operating Model Conditioning]~
    \begin{itemize}   
        \item The OM has to have prediction skill? 
        \item The OEM has to be developed as part of the OM. Therefore compare runs tests across scenarios? i.e. are there alternative equally plausible hypotheses? If so are there potential problems with the OEM, since the MP has to use the same pseudo data what if for an empirical HCR the best index depends on the OM?
    \item After fitting, other questions are grid search or random search? and how to model future processes such as recruitment, selectivity, and catchability? To avoid answering these questions in the 1st stage we recommend the use of backtesting, i.e. extending the model based hindcast to include feedback (Next paper).
  \end{itemize}

    \item[Other issues include]~
    \begin{itemize}
        \item Estimation v model error
        \item Emergent properties, is it the choice of OM scenario less important than the properties of the OM. For example in an OM the similar values for reference points, the form of the production function, current status, and future variability can be generated by different OM hypotheses. While, the performance of a black box controller will be dictated by factors such as trends in the index and system properties such as population doubling time and the level of process error. 
        %\item Use LASSO to identify what matters.
        \item %The nature of the indices is also important; for example even if a stock had low inter-annual variability, an index could be highly variable if it was based on juveniles or there were large changes in spatial distribution between years. 
        It is necessary to look at the robustness of management strategies to the nature of the time series of the stock (as represented by the OM) and to the characteristics of the data collected from it. This will require tuning by constructing a reference set of OMs and then tuning the management strategy to secure the desired trade-offs. The work so far can be considered as focusing first on developing management strategies that perform satisfactorily for a reference set; the next step is to develop case-specific strategies.
\end{itemize}
\end{description}
